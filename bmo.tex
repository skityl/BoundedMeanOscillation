\documentclass{unswmaths}

\usepackage{unswshortcuts}

\begin{document}

\subject{}
\author{}
\title{}
\studentno{}


\newcommand{\Real}{\operatorname{Re}}
\newcommand{\Img}{\operatorname{Im}}
\newcommand{\lan}{\langle}
\newcommand{\ran}{\rangle}
\newcommand{\Proj}{\mathbb{P}}
\newcommand{\isom}{\cong}
\newcommand{\id}{{\operatorname{id}}}
\newcommand{\ha}{\mathbf{m}}
\newcommand{\Circ}{\mathbb{T}}
\newcommand{\BMO}{{BMO}}
\newcommand{\sgn}{\operatorname{sgn}}
\newcommand{\Diff}{\mathcal{D}}


We consider functions on the circle $\Circ$, equipped
with normalised Haar measure $\ha$.

We denote the fourier transform $L^2(\circ,\ha) \rightarrow \ell^2(\Intgr)$ 
by $\wedge$, and the inverse fourier transform by $\vee$. Explicitly,
for $f \in L^2(\Circ)$, 
\begin{equation*}
    \hat{f}(n) = \int_{\Circ} f(\zeta)\zeta^{-n}\;d\ha(\zeta),
\end{equation*}
and for $s \in \ell^2(\Intgr)$,
\begin{equation*}
    \check{s}(z) = \sum_{k \in \Intgr} s(k)z^k.
\end{equation*}
Recall the standard result that this is an isometric isomorphism.

We adopt the following notation for sequences. For $s \in \ell^2(\Intgr)$, write
\begin{equation*}
    s = (s(n)\;:\;n \in \Intgr).
\end{equation*}

Several important linear operators on $L^2(\Circ)$ can be defined by the Fourier
transform. For example by the \emph{harmonic conjugate} $\tilde{f}$ of a $f \in L^2(\Circ)$, we mean
\begin{equation*}
    \tilde{f} = (\sgn(n)\hat{f}(n):n \in \Intgr)^{\vee}.
\end{equation*}
The Riesz projection, $\Proj_+$ and $\Proj_-$ are defined respectively for $f \in L^2(\Circ)$ by
\begin{align*}
        \widehat{\Proj_+ f}(n) &= \begin{cases} \hat{f}(n)\;\text{ if }n \geq 0\\
        0\text{ otherwise.}\\
   \end{cases}\\
   \widehat{\Proj_- f}(n) &= \begin{cases} 0\text{ if }n \geq 0\\
        \hat{f}(n)\text{ otherwise.}\\
   \end{cases}
\end{align*}

Clearly the harmonic conjugation and Riesz projection operators are bounded. We may similarly
define unbounded operators in the same way. For example, the differentiation operator $\Diff$ is given by
\begin{equation*}
    \Diff f = (n\hat{f}(n)\;:\;n \in \Intgr)^\vee.
\end{equation*}

Given $\varphi \in L^2(\Circ)$, we define $\Gamma_\varphi$ as the operator
\begin{equation*}
    \Gamma_\varphi f = (\sum_{k=0}^\infty \hat{\varphi}(k+n)\hat{f}(k)\;:\;n \in \Intgr)^\vee.
\end{equation*}
$\Gamma_\varphi$ is potentially unbounded. We have the following theorem which characterises
those $\varphi$ for which $\Gamma_\varphi$ is bounded. 

\begin{theorem}
    $\Gamma_\varphi$ is bounded if and only if $\varphi \in L^\infty(\Circ)$.
\end{theorem}
\begin{proof}
    Let $f \in L^2(\Circ)$. Since the fourier transform is isometric, we may compute the norm of $\Gamma_\varphi f$,
    \begin{equation*}
        \|\Gamma_\varphi f\|^2 = \sum_{
    \end{equation*}
\end{proof}

\begin{definition}
    The set $\BMO(\Circ)$ is given by
    \begin{equation*}
        BMO(\Circ) = \{\;f+\tilde{g}\;:\;f,g \in L^\infty(\Circ)\;\} \subset L^2(\Circ).
    \end{equation*}
\end{definition}



\end{document}