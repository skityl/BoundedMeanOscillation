\documentclass{unswmaths}

\usepackage{unswshortcuts}

\begin{document}

\subject{}
\author{}
\title{}
\studentno{}


\newcommand{\Real}{\operatorname{Re}}
\newcommand{\Img}{\operatorname{Im}}
\newcommand{\lan}{\langle}
\newcommand{\ran}{\rangle}
\newcommand{\Proj}{\mathbb{P}}
\newcommand{\isom}{\cong}
\newcommand{\id}{{\operatorname{id}}}
\newcommand{\ha}{\boldsymbol{m}}
\newcommand{\Circ}{\mathbb{T}}
\newcommand{\BMO}{{BMO}}
\newcommand{\sgn}{\operatorname{sgn}}
\newcommand{\Diff}{\mathcal{D}}
\newcommand{\pvint}{\mathrm{p.v.}\int}

\section*{Introduction}
The purpose of these notes is to give a complete proof of the characterisation of functions
$f$ on the circle such that the quantised derivative $df$ is bounded. The quantised derivative
is introduced in Connes \cite{connes94}. These notes use ideas and notation from Peller \cite{peller}.

\section*{Notation} 
We consider functions on the circle $\Circ = \{z \in \Cplx\;:\;|z| = 1\}$, equipped
with normalised Haar measure $\ha$.

The space $L^2(\Circ,\ha)$ as usual is the Hilbert space of square-integrable functions
on $\Circ$ equipped with the inner product
\begin{equation*}
    \langle f,g\rangle = \int_{\Circ} f(\zeta)\overline{g(\zeta)}\;d\ha(\zeta).
\end{equation*}
for $f,g \in L^2(\Circ, \ha)$.

For convenience, we will denote $L^p(\Circ,\ha)$ as $L^p(\Circ)$.


Sequences indexed by a set $S$ will be denoted in parentheses as
\begin{equation*}
    (s_k\;:\;k \in S).
\end{equation*}

We denote the Fourier transform $L^2(\Circ,\ha) \rightarrow \ell^2(\Intgr)$ 
by $\wedge$, and the inverse Fourier transform by $\vee$. Explicitly,
for $f \in L^2(\Circ)$, 
\begin{equation*}
    \hat{f}(n) = \int_{\Circ} f(\tau)\tau^{-n}\;d\ha(\tau),
\end{equation*}
and for $s \in \ell^2(\Intgr)$,
\begin{equation*}
    \check{s}(z) = \sum_{k \in \Intgr} s(k)z^k.
\end{equation*}
Recall the standard result that this is an isometric isomorphism of Hilbert spaces.

So for example, in our notation the inverse Fourier transform of a sequence
\begin{equation*}
    (s_k\;:\;k \in \Intgr)
\end{equation*}
is denoted
\begin{equation*}
    (s_k\;:\;k \in \Intgr)^\vee.
\end{equation*}

If we have a sequence $(s_k\;:\;k \in \Ntrl)$ indexed by $\Ntrl$, interpret its
inverse Fourier transform as
\begin{equation*}
    (s_k\;:\;k \in \Ntrl)^\vee(z) = \sum_{k=0}^\infty s_kz^k.
\end{equation*}

Several important linear operators on $L^2(\Circ)$ can be defined by the Fourier
transform. For example define the \emph{harmonic conjugate} $\tilde{f}$ of $f \in L^2(\Circ)$ as
\begin{equation*}
    \tilde{f} := (\sgn(n)\hat{f}(n):n \in \Intgr)^{\vee}.
\end{equation*}
where $\sgn(n) = -1$ if $n < 0$ and $\sgn(n) = 1$ otherwise.


The Riesz projections, $\Proj_+$ and $\Proj_-$ are defined respectively for $f \in L^2(\Circ)$ by
\begin{align*}
        \widehat{\Proj_+ f}(n) &= \begin{cases} \hat{f}(n)\;\text{ if }n \geq 0\\
        0\text{ otherwise.}\\
   \end{cases}\\
   \widehat{\Proj_- f}(n) &= \begin{cases} 0\text{ if }n \geq 0\\
        \hat{f}(n)\text{ otherwise.}\\
   \end{cases}
\end{align*}
Note that $\Proj_++\Proj_- = I$, where $I$ is the identity map.

Clearly the harmonic conjugation and Riesz projection operators are bounded. We may
define unbounded operators similarly. For example, the differentiation operator $\Diff$ is given by
\begin{equation*}
    \Diff f = (n\hat{f}(n)\;:\;n \in \Intgr)^\vee.
\end{equation*}

We also define the important \emph{Hilbert transform}, $H$, as
\begin{equation*}
    H = \frac{1}{2}(\Proj_+-\Proj_-) = \Proj_+-\frac{1}{2}I.
\end{equation*}

For $\varphi \in L^2(\Circ)$, the point-wise multiplication operator $M_\varphi:L^2(\Circ)\rightarrow L^2(\Circ)$
is
\begin{equation*}
    M_\varphi f(z) = \varphi(z)f(z).
\end{equation*}
for almost all $z \in \Circ$ and $f \in L^2(\Circ)$. $M_\varphi$ is bounded and defined everywhere if $\varphi \in L^\infty(\Circ)$, however
otherwise $M_\varphi$ may be unbounded or not everywhere defined. $M_\varphi$ is at least defined on the dense subspace
of polynomials in $L^2(\Circ)$. 

\section*{The Quantised Derivative}

\begin{definition}
    For $\varphi \in L^2(\Circ)$, the \emph{quantised derivative} of $\varphi$ is the (potentially unbounded, densely defined) linear operator
    \begin{equation*}
        d\varphi := [H,M_\varphi] = [\Proj_+,M_\varphi].
    \end{equation*}
\end{definition}
The purpose of these notes is to characterise the boundedness of $d\varphi$ in terms of $\varphi$.

We require an alternative description of $d\varphi$. For this, we introduce the Hardy spaces.
\begin{definition}
    For $p\geq 2$, define
    \begin{align*}
        H^p(\Circ) := \Proj_+ L^p(\Circ)\\
        H^p_-(\Circ) := \Proj_- L^p(\Circ).
    \end{align*}
    So that $L^2(\Circ) = H^2(\Circ)\oplus H^2_-(\Circ)$. 
\end{definition}

Hence, we may consider the quantised derivative $d\varphi$ as a function from $H^2(\Circ)\oplus H^2_-(\Circ)$.

\begin{lemma}
    Let $\varphi \in L^2(\Circ)$ and define $\varphi_+ := \Proj_+ \varphi$ and $\varphi_- := \Proj_- \varphi$. Then $d\varphi:H^2(\Circ)\oplus H^2_-(\Circ)\rightarrow H^2(\Circ)\oplus H^2_-(\Circ)$ may be written as
    \begin{equation*}
        d\varphi(f\oplus g) = 
            (\Proj_+M_{\varphi_+})g \oplus
            -(\Proj_-M_{\varphi_-})f
    \end{equation*}
    for $f \in H^2(\Circ)$ and $g \in H^2_-(\Circ)$.
\end{lemma}
\begin{proof}
    This is a simple computation. Let $f \in H^2(\Circ)$ and $g \in H^2_-(\Circ)$. Then,
    \begin{equation*}
        d\varphi(f+g) = [\Proj_+,M_{\varphi_+}+M_{\varphi_-}](f+g).
    \end{equation*}
    Hence,
    \begin{align*}
        d\varphi(f+g) &= [\Proj_+,M_{\varphi_+}]f + [\Proj_+,M_{\varphi_-}]f + [\Proj_+,M_{\varphi_+}]g + [\Proj_+,M_{\varphi_-}]g\\
        &= (\Proj_+M_{\varphi_-})f+(\Proj_+M_{\varphi_+})g-M_{\varphi_-}f
    \end{align*}
    since $\Proj_+f = f$ and $\Proj_+g = 0$.
    
    By the identity $\Proj_+ = I-\Proj_-$, we find
    \begin{equation*}
        d\varphi(f+g) = (\Proj_+M_{\varphi_+})g - (\Proj_-M_{\varphi_-})f.
    \end{equation*}
\end{proof}

The problem of determining the boundedness of $d\varphi$ is then reduced to the problem of determining
the boundedness of operators of the form $\Proj_+M_\psi:H^2_-(\Circ)\rightarrow H^2(\Circ)$ and $\Proj_-M_\psi:H^2(\Circ)\rightarrow H^2_-(\Circ)$ for $\psi \in L^2(\Circ)$. We may simplify
this further with the following lemma:
\begin{lemma}
    Let $\psi \in L^2(\Circ)$. Then
    \begin{equation*}
        (\Proj_+M_\psi)^* = \Proj_-M_{\overline{\psi}}.
    \end{equation*}
    and therefore $\Proj_+M_\psi$ is bounded if and only if $\Proj_-M_{\overline{\psi}}$ is.
\end{lemma}
\begin{proof}
    Let $e_k(z) = z^k$. 

    This is again a simple computation. Let $m,n\in \Intgr$ with $m \geq 0$ and $n < 0$. Then,
    \begin{align*}
        \langle (\Proj_+M_\psi)e_n,e_m\rangle &= \int_{\Circ}\sum_{k>-n} \hat{\psi}(k)\zeta^{k+n-m}\;d\ha(\zeta)\\
        &= \hat{\psi}(m-n).
    \end{align*}
    
    Similarly,
    \begin{align*}
        \langle e_n, (\Proj_-M_{\overline{\psi}})e_m \rangle &= \int_{\Circ} \sum_{k > m} \hat{\varphi}(k) \zeta^{n-m+k}\;d\ha(\zeta)\\
                                                &= \hat{\psi}(m-n).
    \end{align*}
    
    Hence, $(\Proj_+M_\psi)^* = \Proj_-M_{\overline{\psi}}$.

\end{proof}

Therefore, we only need to study operators of the form $\Proj_-M_\psi$. So for $\psi \in L^2(\Circ)$, define
\begin{equation*}
    H_\psi := \Proj_-M_\psi:H^2\rightarrow H^2_-.
\end{equation*}

We study these operators using the Fourier transform. Use the standard basis $\{z^n\}_{n\geq 0}$
on $H^2(\Circ)$ and the standard basis with negative indices $\{z^{-n}\}_{n \geq 0}$ on $H^2(\Circ)$.

Let $\psi \in L^2(\Circ)$. Then in the bases above, $H_\psi$ has matrix representation
with $(n,k)$th entry $\hat{\psi}(-n-k)$. 

This means that $H_\psi$ is represented by a \emph{Hankel matrix}. So we require
results on the boundedness of Hankel matrices



\begin{definition}
    The set $\BMO(\Circ)$ is given by
    \begin{equation*}
        BMO(\Circ) = \{\;f+\tilde{g}\;:\;f,g \in L^\infty(\Circ)\;\} \subset L^2(\Circ).
    \end{equation*}
    $\BMO(\Circ)$ is a vector space since harmonic conjugation is linear.
\end{definition}
$\BMO(\Circ)$ is usually given a different definition, for example as in \cite[p.~216]{garnett}.
For a proof of the equivalence of this definition, see \cite[p.~240]{garnett}.


\begin{lemma}
    $\varphi \in \BMO(\Circ)$ if and only if $\Proj_+\varphi$ and $\Proj_-\varphi$ are in $\BMO(\Circ)$.
\end{lemma}
\begin{proof}
    If $\Proj_+\varphi$ and $\Proj_-\varphi$ are in $\BMO(\Circ)$, then $\varphi = \Proj_+\varphi +\Proj_-\varphi$ is
    in $\BMO(\Circ)$.
    
    Conversely, suppose $\varphi \in \BMO(\Circ)$. Then there are $f,g \in L^\infty(\Circ)$ such that
    \begin{equation*}
        \varphi = f + \tilde{g}.
    \end{equation*}
    Hence,
    \begin{equation*}
        \Proj_+\varphi = \Proj_+(f+g) = \frac{1}{2}(f+g+\tilde{f}+\tilde{g}) \in \BMO(\Circ).
    \end{equation*}
    Similarly,
    \begin{equation*}
        \Proj_+\varphi = \Proj_-(f-g) = \frac{1}{2}(f-g-\tilde{f}+\tilde{g}) \in \BMO(\Circ).
    \end{equation*}
\end{proof}


\section*{Infinite Hankel matrices}
Let $s = (s_k\;:\;k \in \Ntrl)$ be an infinite sequence. The associated
Hankel matrix is the infinite matrix $\Gamma_s$ with $(j,k)$th entry
$s_{k+j}$. We may consider such a matrix as as an operator
on the dense subset of finitely supported sequences in $\ell^2(\Ntrl)$. We will investigate conditions
for $\Gamma_s$ to be bounded, and therefore extend to a linear operator on $\ell^2(\Ntrl)$.

\begin{theorem}
    The operator $\Gamma_s$ is bounded if and only if $\check{s} \in \BMO(\Circ)$.
\end{theorem}
\begin{proof}
    Suppose first that $\check{s} \in L^\infty(\Circ)$. Let $a = (a_k\;:\;k \in \Ntrl)$
    and $b = (b_k\;:\; k\in \Ntrl)$ be finitely supported sequences. Let $q(z) = \check{a}(z)\check{\overline{b}}(z)$. Then,
    \begin{align*}
        \langle \Gamma_s a,b\rangle  &=  \sum_{j,k\geq0} s_{j+k}a_j\overline{b_k}\\
        &= \sum_{m\geq 0} s_m \sum_{j=0}^m a_j \overline{b_{m-j}}\\
        &= \sum_{m\geq 0} s_m \hat{q}(m)\\
        &= \int_\Circ \check{s}(\tau)q(\overline{\tau})\;d\ha(\tau).
    \end{align*}
    Hence,
    \begin{equation*}
        |\langle \Gamma_s a,b\rangle| \leq \|\check{s}\|_\infty \|q\|_1 \leq \|\check{s}\|_\infty \|\check{a}\|_{L^2}\|\check{b}\|_{L^2} = \|\check{s}\|_\infty \|a\|_{\ell^2}\|b\|_{\ell^2}.
    \end{equation*}
    Hence $\Gamma_s$ is bounded. 
    
    Conversely, suppose that $\Gamma_s$ is bounded. 
    
    Define the linear functional $\mathcal{L}$ on the set of polynomials 
    in $H^1(\Circ)$ as
    \begin{equation*}
        \mathcal{L}q = \sum_{n\geq 0} s_n\hat{q}(n).
    \end{equation*}
    If $s \in \ell^1(\Ntrl)$, then $\mathcal{L}$ is bounded by the triangle inequality.
    
\end{proof}

\section*{Integral representation of the quantised derivative}
The following theorem allows us to describe the quantised derivative as an integral operator.
\begin{lemma}
\label{singularIntegral}
    \begin{equation*}
        \mathrm{p.v.}\int_\Circ \frac{1}{\tau-1}\;d\ha(\tau) = -\frac{1}{2}
    \end{equation*}
    where the principal value is defined to be
    \begin{equation*}
        \lim_{\varepsilon\rightarrow 0} \int_{|\tau-1|>\varepsilon} \frac{1}{\tau-1}\;d\ha(\tau).
    \end{equation*}
\end{lemma} 
\begin{proof}
    Note that
    \begin{equation*}
        \pvint_\Circ \frac{1}{\tau-1}\;d\ha(\tau) = \pvint_{\im(\tau) > 0} \frac{1}{\overline{\tau}-1}+\frac{1}{\tau-1}\;d\ha(\tau).
    \end{equation*}
    Hence,
    \begin{equation*}
        \pvint_\Circ \frac{1}{\tau-1}\;d\ha(\tau) = \pvint_{\operatorname{Im}(\tau)>0} 2\operatorname{Re}\left(\frac{1}{\tau-1}\right)\;d\ha(\tau).
    \end{equation*}
    However, if $\tau = \exp(i\theta) \neq 1$, then
    \begin{align*}
        \operatorname{Re}\left(\frac{1}{\tau-1}\right) &= \operatorname{Re}\left(\frac{e^{-i\theta/2}}{2i\sin(\theta/2)}\right)\\
        &= -\frac{1}{2}.
    \end{align*}
    Hence,
    \begin{equation*}
        \pvint_\Circ \frac{1}{\tau-1}\;d\ha(\tau) = 2\pvint_{\operatorname{Im}(\tau)>0} -\frac{1}{2}\;d\ha(\tau) = -\frac{1}{2}.
    \end{equation*}
\end{proof}
\begin{theorem}
    Let $\varphi \in L^2(\Circ)$. Then
    \begin{equation*}
        \Proj_+\varphi(z) = \mathrm{p.v.}\int_\Circ \frac{\varphi(\tau)}{1-\overline{\tau}z}\;d\ha(\tau)+\frac{1}{2}\varphi(z)
    \end{equation*}
    and hence,
    \begin{equation*}
        H\varphi(z) = \mathrm{p.v.}\int_\Circ \frac{\varphi(\tau)}{1-\overline{\tau}z}\;d\ha(\tau).
    \end{equation*}
    where in both equations, the principal value means that the integral is to be taken along the set $\{ \zeta\;:|\zeta-z| > \varepsilon\}$
    and then consider the limit $\varepsilon\rightarrow 0$.
\end{theorem}
\begin{proof}
    It is sufficient to check this on the basis elements $e_n(z) = z^n$ for $n \in \Intgr$.
    
    First let $n \geq 0$. Then
    \begin{equation*}
        \mathrm{p.v.}\int_\Circ \frac{\tau^n}{1-\overline{\tau}z}\;d\ha(\tau) = \mathrm{p.v.}\int_\Circ \frac{z^n\tau^n}{1-\overline{\tau}}\;d\ha(\tau)
    \end{equation*}
    by translation invariance.
    Hence,
    \begin{align*}
        \mathrm{p.v.}\int_{\Circ} \frac{\tau^n}{1-\overline{\tau}z}\;d\ha(\tau) &= z^n\mathrm{p.v.}\int_\Circ \frac{\tau^{n+1}}{\tau-1}\;d\ha(\tau) \\
        &= z^n \mathrm{p.v.}\int_\Circ \frac{\tau^{n+1}-1}{\tau-1}+\frac{1}{\tau-1}\;d\ha(\tau)\\
        &= z^n \mathrm{p.v.}\int_\Circ 1+\tau+\tau^2+\cdots+\tau^{n}\;d\ha(\tau)+z^n\mathrm{p.v.}\int_\Circ \frac{1}{\tau-1}\;d\ha(\tau)\\
        &= z^n + z^n\mathrm{p.v.}\int_\Circ \frac{1}{\tau-1}\;d\ha(\tau)\\
        &= \frac{1}{2}z^n
    \end{align*}
    where the last step follows from lemma \ref{singularIntegral}.
    
    Suppose $n > 0$, then
    \begin{equation*}
        \mathrm{p.v.}\int_\Circ \frac{\tau^{-n}}{1-\overline{\tau}z}\;d\ha(\tau) = z^{-n} \mathrm{p.v.}\int_{\Circ} \frac{\tau{1-n}}{\tau-1}\;d\ha(\tau)
    \end{equation*}
    by translation invariance. Hence,
    \begin{align*}
        \mathrm{p.v.}\int_\Circ \frac{\tau^{-n}}{1-\overline{\tau}z}\;d\ha(\tau) &= z^{-n} \mathrm{p.v.} \int_\Circ \frac{1}{\tau^n-\tau^{n-1}}\;d\ha(\tau)\\
        &= z^{-n}\overline{\mathrm{p.v.}\int_\Circ \frac{\tau^n}{1-\tau}}\\
        &= -\frac{1}{2}z^{-n}.
    \end{align*}
    
    Hence, 
    \begin{equation*}
        \pvint_\Circ \frac{\tau^n}{1-\overline{\tau}z}\;d\ha(\tau) = \begin{cases}
            \frac{1}{2}z^n\text{ if }n \geq 0\\
            -\frac{1}{2}z^n\text{ if }n < 0.
        \end{cases}
    \end{equation*}
    
    So the result follows.
    
\end{proof}

So we have the following integral form of the quantised derivative. Let $\varphi,f \in L^2(\Circ)$.
Then
\begin{equation*}
    d\varphi(f)(z) = ([H,M_\varphi]f)(z) = \pvint_\Circ \frac{\varphi(z)-\varphi(\tau)}{1-\overline{\tau}z}f(\tau)\;d\ha(\tau).
\end{equation*}

\begin{thebibliography}{9}
\bibitem{connes94}
     Connes A., 
    \emph{Noncommutative Geometry}
     Academic Press, 
     San Diego, 
     CA, 
     1994
\bibitem{peller}
    Peller V.V.,
    \emph{Hankel Operators and their Applications}
    Springer-Verlag,
    New York, 
    NY,
    2003    
\bibitem{garnett}
    Garnett J.B.
    \emph{Bounded analytic functions}
    Springer-Verlag,
    New York,
    NY,
    2007    
\end{thebibliography}


\end{document}