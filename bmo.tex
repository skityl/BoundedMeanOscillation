\documentclass{unswmaths}

\usepackage{unswshortcuts}

\begin{document}

\subject{}
\author{}
\title{}
\studentno{}


\newcommand{\Real}{\operatorname{Re}}
\newcommand{\Img}{\operatorname{Im}}
\newcommand{\lan}{\langle}
\newcommand{\ran}{\rangle}
\newcommand{\Proj}{\mathbb{P}}
\newcommand{\isom}{\cong}
\newcommand{\id}{{\operatorname{id}}}
\newcommand{\ha}{\boldsymbol{m}}
\newcommand{\Circ}{\mathbb{T}}
\newcommand{\BMO}{{BMO}}
\newcommand{\sgn}{\operatorname{sgn}}
\newcommand{\Diff}{\mathcal{D}}

\section*{Introduction}
The purpose of these notes is to give a complete proof of the characterisation of functions
$f$ on the circle such that the quantised derivative $df$ is bounded. The quantised derivative
is introduced in Connes \cite{connes94}. These notes use ideas and notation from Peller \cite{peller}.

\section*{Notation} 
We consider functions on the circle $\Circ = \{z \in \Cplx\;:\;|z| = 1\}$, equipped
with normalised Haar measure $\ha$.

The space $L^2(\Circ,\ha)$ as usual is the hilbert space of square-integrable functions
on $\Circ$ equipped with the inner product
\begin{equation*}
    \langle f,g\rangle = \int_{\Circ} f(\zeta)\overline{g(\zeta)}\;d\ha(\zeta).
\end{equation*}
for $f,g \in L^2(\Circ, \ha)$.

For convenience, we will denote $L^2(\Circ,\ha)$ as $L^p(\Circ)$.


Sequences indexed by a set $S$ will be denoted in parentheses as
\begin{equation*}
    (s_k\;:\;k \in S).
\end{equation*}

We denote the fourier transform $L^2(\Circ,\ha) \rightarrow \ell^2(\Intgr)$ 
by $\wedge$, and the inverse fourier transform by $\vee$. Explicitly,
for $f \in L^2(\Circ)$, 
\begin{equation*}
    \hat{f}(n) = \int_{\Circ} f(\zeta)\zeta^{-n}\;d\ha(\zeta),
\end{equation*}
and for $s \in \ell^2(\Intgr)$,
\begin{equation*}
    \check{s}(z) = \sum_{k \in \Intgr} s(k)z^k.
\end{equation*}
Recall the standard result that this is an isometric isomorphism of Hilbert spaces.

So for example, in our notation the inverse fourier transform of a sequence
\begin{equation*}
    (s_k\;:\;k \in S)
\end{equation*}
is denoted
\begin{equation*}
    (s_k\;:\;k \in S)^\vee.
\end{equation*}

Several important linear operators on $L^2(\Circ)$ can be defined by the Fourier
transform. For example define the \emph{harmonic conjugate} $\tilde{f}$ of $f \in L^2(\Circ)$ as
\begin{equation*}
    \tilde{f} := (\sgn(n)\hat{f}(n):n \in \Intgr)^{\vee}.
\end{equation*}
where $\sgn(n) = -1$ if $n < 0$ and $\sgn(n) = 1$ otherwise.


The Riesz projections, $\Proj_+$ and $\Proj_-$ are defined respectively for $f \in L^2(\Circ)$ by
\begin{align*}
        \widehat{\Proj_+ f}(n) &= \begin{cases} \hat{f}(n)\;\text{ if }n \geq 0\\
        0\text{ otherwise.}\\
   \end{cases}\\
   \widehat{\Proj_- f}(n) &= \begin{cases} 0\text{ if }n \geq 0\\
        \hat{f}(n)\text{ otherwise.}\\
   \end{cases}
\end{align*}

Clearly the harmonic conjugation and Riesz projection operators are bounded. We may similarly
define unbounded operators in the same way. For example, the differentiation operator $\Diff$ is given by
\begin{equation*}
    \Diff f = (n\hat{f}(n)\;:\;n \in \Intgr)^\vee.
\end{equation*}

We also define the important \emph{Hilbert transform}, $H$, as
\begin{equation*}
    H = \frac{1}{2}(\Proj_+-\Proj_-) = \Proj_+-\frac{1}{2}I.
\end{equation*}
where $I$ is the identity map.

For $\varphi \in L^2(\Circ)$, the pointwise multiplication operator $M_\varphi:L^2(\Circ)\rightarrow L^2(\Circ)$
is
\begin{equation*}
    M_\varphi f(z) = \varphi(z)f(z).
\end{equation*}
for almost all $z \in \Circ$ and $f \in L^2(\Circ)$. $M_\varphi$ is bounded and defined everywhere if $\varphi \in L^\infty(\Circ)$, however
otherwise $M_\varphi$ may be unbounded or not everywhere defined. $M_\varphi$ is at least defined on the dense subspace
of polynomials in $L^2(\Circ)$. 

\section*{The Quantised Derivative}

\begin{definition}
    For $\varphi \in L^2(\Circ)$, the \emph{quantised derivative} of $\varphi$ is the (potentially unbounded, densely defined) linear operator
    \begin{equation*}
        d\varphi := [H,M_\varphi] = [\Proj_+,M_\varphi].
    \end{equation*}
\end{definition}
The purpose of these notes is to characterise the boundedness of $d\varphi$ in terms of $\varphi$.

We require an alternative description of $d\varphi$. For this, we introduce the Hardy spaces.
\begin{definition}
    For $p\geq 2$, define
    \begin{align*}
        H^p(\Circ) := \Proj_+ L^p(\Circ)\\
        H^p_-(\Circ) := \Proj_- L^p(\Circ).
    \end{align*}
    So that $L^2(\Circ) = H^2(\Circ)\oplus H^2_-(\Circ)$. 
\end{definition}

Hence, we may consider the quantised derivative $d\varphi$ as a function from $H^2(\Circ)\oplus H^2_-(\Circ)$.

\begin{lemma}
    Let $\varphi \in L^2(\Circ)$ and define $\varphi_+ := \Proj_+ \varphi$ and $\varphi_- := \Proj_- \varphi$. Then $d\varphi:H^2(\Circ)\oplus H^2_-(\Circ)\rightarrow L^2(\Circ)$ may be written as
    \begin{equation*}
        d\varphi\begin{pmatrix}
            f\\g
        \end{pmatrix} = \begin{pmatrix}
            (\Proj_+M_{\varphi_+})g\\
            -(\Proj_-M_{\varphi_-})f
        \end{pmatrix}
    \end{equation*}
    for $f \in H^2(\Circ)$ and $g \in H^2_-(\Circ)$.
\end{lemma}
\begin{proof}
    This is a simple computation. Let $f \in H^2(\Circ)$ and $g \in H^2_-(\Circ)$. Then,
    \begin{equation*}
        d\varphi(f+g) = [\Proj_+,M_{\varphi_+}+M_{\varphi_-}](f+g).
    \end{equation*}
    Hence,
    \begin{align*}
        d\varphi(f+g) &= [\Proj_+,M_{\varphi_+}]f + [\Proj_+,M_{\varphi_-}]f + [\Proj_+,M_{\varphi_+}]g + [\Proj_+,M_{\varphi_-}]g\\
        &= (\Proj_+M_{\varphi_-})f+(\Proj_+M_{\varphi_+})g-M_{\varphi_-}f
    \end{align*}
    since $\Proj_+f = f$ and $\Proj_+g = 0$.
    
    By the identity $\Proj_+ = I-\Proj_-$, we find
    \begin{equation*}
        d\varphi(f+g) = (\Proj_+M_{\varphi_+})g - (\Proj_-M_{\varphi_-})f.
    \end{equation*}
\end{proof}

The problem of determining the boundedness of $d\varphi$ is then reduced to the problem of determining
the boundedness of operators of the form $\Proj_+M_\psi$ and $\Proj_-M_\psi$ for $\psi \in L^2(\Circ)$. We may simplify
this further with the following lemma:
\begin{lemma}
    Let $\psi \in L^2(\Circ)$. Then
    \begin{equation*}
        (\Proj_+M_\psi)^* = \Proj_-M_{\overline{\psi}}.
    \end{equation*}
    and therefore $\Proj_+M_\psi$ is bounded if and only if $\Proj_-M_{\overline{\psi}}$ is.
\end{lemma}
\begin{proof}
    Let $e_k(z) = z^k$. 

    This is again a simple computation. Let $m,n\in \Intgr$ with $m \geq 0$. Then
    \begin{align*}
        \langle (\Proj_+M_\psi)z^n,z^m\rangle &= \int_{\Circ}\sum_{k=-n}^\infty \hat{\psi}(k)\zeta^{k+n-m}\;d\ha(\zeta)\\
        &= \hat{\psi}(m-n)
    \end{align*}
    and if $m < 0$, then $\langle (\Proj_+M_\psi)z^n,z_m\rangle = 0$.
    
    Similarly, if $m \geq 0$,
    \begin{align*}
        \langle z^n, (\Proj_-M_{\overline{\psi})z^m\rangle &= \int_{\Circ} \sum_{k=-\infty}^{-m-1} \hat{\varphi}(k) \zeta^{n-m-k}\;d\ha(\zeta)\\
        &= \psi(m-n).
    \end{align*}
    
\end{proof}


Given $\varphi \in L^2(\Circ)$, we define $\Gamma_\varphi$ as the operator
\begin{equation*}
    \Gamma_\varphi f = (\sum_{k=0}^\infty \hat{\varphi}(k+n)\hat{f}(k)\;:\;n \in \Intgr)^\vee.
\end{equation*}
$\Gamma_\varphi$ is potentially unbounded. We have the following theorem which characterises
those $\varphi$ for which $\Gamma_\varphi$ is bounded. 
\begin{theorem}
    $\Gamma_\varphi$ is bounded if and only if $\varphi \in L^\infty(\Circ)$.
\end{theorem}
\begin{proof}
    Let $f \in L^2(\Circ)$. Since the fourier transform is isometric, we may compute the norm of $\Gamma_\varphi f$,
    \begin{equation*}
%        \|\Gamma_\varphi f\|^2 = \sum_{
    \end{equation*}
\end{proof}

\begin{definition}
    The set $\BMO(\Circ)$ is given by
    \begin{equation*}
        BMO(\Circ) = \{\;f+\tilde{g}\;:\;f,g \in L^\infty(\Circ)\;\} \subset L^2(\Circ).
    \end{equation*}
\end{definition}


\begin{thebibliography}{9}
\bibitem{connes94}
    Alain Connes, 
    \emph{Noncommutative Geometry}
     Academic Press, 
     San Diego, 
     CA, 
     1994
\bibitem{peller}
    Vladimir Peller,
    \emph{Hankel Operators and their Applications}
    Springer-Verlag,
    New York, 
    NY,
    2003
    
\end{thebibliography}


\end{document}